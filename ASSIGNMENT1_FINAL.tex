\documentclass[]{article}
\usepackage{lmodern}
\usepackage{amssymb,amsmath}
\usepackage{ifxetex,ifluatex}
\usepackage{fixltx2e} % provides \textsubscript
\ifnum 0\ifxetex 1\fi\ifluatex 1\fi=0 % if pdftex
  \usepackage[T1]{fontenc}
  \usepackage[utf8]{inputenc}
\else % if luatex or xelatex
  \ifxetex
    \usepackage{mathspec}
  \else
    \usepackage{fontspec}
  \fi
  \defaultfontfeatures{Ligatures=TeX,Scale=MatchLowercase}
\fi
% use upquote if available, for straight quotes in verbatim environments
\IfFileExists{upquote.sty}{\usepackage{upquote}}{}
% use microtype if available
\IfFileExists{microtype.sty}{%
\usepackage{microtype}
\UseMicrotypeSet[protrusion]{basicmath} % disable protrusion for tt fonts
}{}
\usepackage[margin=1in]{geometry}
\usepackage{hyperref}
\hypersetup{unicode=true,
            pdftitle={ESM\_201\_Assignment\_1},
            pdfauthor={Sandra Fogg},
            pdfborder={0 0 0},
            breaklinks=true}
\urlstyle{same}  % don't use monospace font for urls
\usepackage{graphicx,grffile}
\makeatletter
\def\maxwidth{\ifdim\Gin@nat@width>\linewidth\linewidth\else\Gin@nat@width\fi}
\def\maxheight{\ifdim\Gin@nat@height>\textheight\textheight\else\Gin@nat@height\fi}
\makeatother
% Scale images if necessary, so that they will not overflow the page
% margins by default, and it is still possible to overwrite the defaults
% using explicit options in \includegraphics[width, height, ...]{}
\setkeys{Gin}{width=\maxwidth,height=\maxheight,keepaspectratio}
\IfFileExists{parskip.sty}{%
\usepackage{parskip}
}{% else
\setlength{\parindent}{0pt}
\setlength{\parskip}{6pt plus 2pt minus 1pt}
}
\setlength{\emergencystretch}{3em}  % prevent overfull lines
\providecommand{\tightlist}{%
  \setlength{\itemsep}{0pt}\setlength{\parskip}{0pt}}
\setcounter{secnumdepth}{0}
% Redefines (sub)paragraphs to behave more like sections
\ifx\paragraph\undefined\else
\let\oldparagraph\paragraph
\renewcommand{\paragraph}[1]{\oldparagraph{#1}\mbox{}}
\fi
\ifx\subparagraph\undefined\else
\let\oldsubparagraph\subparagraph
\renewcommand{\subparagraph}[1]{\oldsubparagraph{#1}\mbox{}}
\fi

%%% Use protect on footnotes to avoid problems with footnotes in titles
\let\rmarkdownfootnote\footnote%
\def\footnote{\protect\rmarkdownfootnote}

%%% Change title format to be more compact
\usepackage{titling}

% Create subtitle command for use in maketitle
\newcommand{\subtitle}[1]{
  \posttitle{
    \begin{center}\large#1\end{center}
    }
}

\setlength{\droptitle}{-2em}

  \title{ESM\_201\_Assignment\_1}
    \pretitle{\vspace{\droptitle}\centering\huge}
  \posttitle{\par}
    \author{Sandra Fogg}
    \preauthor{\centering\large\emph}
  \postauthor{\par}
      \predate{\centering\large\emph}
  \postdate{\par}
    \date{2/3/2019}

\usepackage{booktabs}
\usepackage{longtable}
\usepackage{array}
\usepackage{multirow}
\usepackage[table]{xcolor}
\usepackage{wrapfig}
\usepackage{float}
\usepackage{colortbl}
\usepackage{pdflscape}
\usepackage{tabu}
\usepackage{threeparttable}
\usepackage{threeparttablex}
\usepackage[normalem]{ulem}
\usepackage{makecell}

\begin{document}
\maketitle

\subsection{Question 1}\label{question-1}

\subsubsection{\texorpdfstring{A. Describe how annual population data
are used to calculate
\(\frac{dN}{Ndt}\).}{A. Describe how annual population data are used to calculate \textbackslash{}frac\{dN\}\{Ndt\}.}}\label{a.-describe-how-annual-population-data-are-used-to-calculate-fracdnndt.}

The expression \(\frac{dN}{Ndt}\) represents the per capita rate change
of a population (r) at time t. By taking the natural log of the annual
population at time (t+1), dividing it by the initial number of
individuals in the same population at time (t), and multiplying it by
the expression \(\frac{1}{(t+1)-t}\), the annual population data can be
used to estimate per capita rate of change of the population. In this
case, the time difference is 1 year, so the second term of the equation
is equal to 1.

\subsubsection{\texorpdfstring{B. Provide the calculation used to
determine \(\frac{dN}{Ndt}\) for each country during the time period
1963 to
1964.}{B. Provide the calculation used to determine \textbackslash{}frac\{dN\}\{Ndt\} for each country during the time period 1963 to 1964.}}\label{b.-provide-the-calculation-used-to-determine-fracdnndt-for-each-country-during-the-time-period-1963-to-1964.}

\subparagraph{\texorpdfstring{Brazil,
\(r= ln\left(\frac{81972001}{79602001}\right)\)}{Brazil, r= ln\textbackslash{}left(\textbackslash{}frac\{81972001\}\{79602001\}\textbackslash{}right)}}\label{brazil-r-lnleftfrac8197200179602001right}

\subparagraph{\texorpdfstring{India,
\(r= ln\left(\frac{486639001}{476632001}\right)\)}{India, r= ln\textbackslash{}left(\textbackslash{}frac\{486639001\}\{476632001\}\textbackslash{}right)}}\label{india-r-lnleftfrac486639001476632001right}

\subparagraph{\texorpdfstring{Japan,
\(r= ln\left(\frac{96959001}{95929001}\right)\)}{Japan, r= ln\textbackslash{}left(\textbackslash{}frac\{96959001\}\{95929001\}\textbackslash{}right)}}\label{japan-r-lnleftfrac9695900195929001right}

\subparagraph{\texorpdfstring{Mexico,
\(r= ln\left(\frac{43052001}{41715001}\right)\)}{Mexico, r= ln\textbackslash{}left(\textbackslash{}frac\{43052001\}\{41715001\}\textbackslash{}right)}}\label{mexico-r-lnleftfrac4305200141715001right}

\subparagraph{\texorpdfstring{South Korea,
\(r= ln\left(\frac{27767001}{27138001}\right)\)}{South Korea, r= ln\textbackslash{}left(\textbackslash{}frac\{27767001\}\{27138001\}\textbackslash{}right)}}\label{south-korea-r-lnleftfrac2776700127138001right}

\begin{table}

\caption{\label{tab:unnamed-chunk-4}Calculated vs. Actual Per Capita Rate of Population Change by Country from 1963 to 1964}
\centering
\begin{tabular}[t]{l|r|r}
\hline
Country & Calculated dN/Ndt & Actual dN/Ndt\\
\hline
Brazil & 0.0293 & 0.0293\\
\hline
India & 0.0208 & 0.0208\\
\hline
Japan & 0.0107 & 0.0107\\
\hline
Mexico & 0.0315 & 0.0315\\
\hline
South Korea & 0.0229 & 0.0229\\
\hline
\end{tabular}
\end{table}

The calculated values are exactly the same as the provided values for
\(\frac{dN}{Ndt}\)

\subsubsection{C. Calculate population rate change of India from 1963 -
2004}\label{c.-calculate-population-rate-change-of-india-from-1963---2004}

\includegraphics{ASSIGNMENT1_FINAL_files/figure-latex/unnamed-chunk-5-1.pdf}

\subsection{QUESTION 2}\label{question-2}

\subsubsection{\texorpdfstring{A. Graph \(\frac{dN}{Ndt}\)
vs.~N\textsubscript{t} for each
country}{A. Graph \textbackslash{}frac\{dN\}\{Ndt\} vs.~Nt for each country}}\label{a.-graph-fracdnndt-vs.nt-for-each-country}

\includegraphics{ASSIGNMENT1_FINAL_files/figure-latex/unnamed-chunk-6-1.pdf}

\subsubsection{B. Estimate the carrying capacity for India and
Brazil}\label{b.-estimate-the-carrying-capacity-for-india-and-brazil}

\paragraph{India}\label{india}

\subparagraph{\texorpdfstring{\(Population = 2.252*10^9 - 7.226*10^{10}(\frac{dN}{Ndt})\)}{Population = 2.252*10\^{}9 - 7.226*10\^{}\{10\}(\textbackslash{}frac\{dN\}\{Ndt\})}}\label{population-2.252109---7.2261010fracdnndt}

\subparagraph{\texorpdfstring{The carrying capacity (K) when
\(\frac{dN}{Ndt}\) = 0 is \(2.252*10^9\) or 2,252,000,000
people.}{The carrying capacity (K) when \textbackslash{}frac\{dN\}\{Ndt\} = 0 is 2.252*10\^{}9 or 2,252,000,000 people.}}\label{the-carrying-capacity-k-when-fracdnndt-0-is-2.252109-or-2252000000-people.}

\paragraph{Brazil}\label{brazil}

\subparagraph{\texorpdfstring{\(Population = 2.607*10^8 - 6.392*10^9(\frac{dN}{Ndt})\)}{Population = 2.607*10\^{}8 - 6.392*10\^{}9(\textbackslash{}frac\{dN\}\{Ndt\})}}\label{population-2.607108---6.392109fracdnndt}

\subparagraph{\texorpdfstring{The carrying capacity (K) when
\(\frac{dN}{Ndt}\) = 0 is \(2.607*10^8\) or 260,700,000
people.}{The carrying capacity (K) when \textbackslash{}frac\{dN\}\{Ndt\} = 0 is 2.607*10\^{}8 or 260,700,000 people.}}\label{the-carrying-capacity-k-when-fracdnndt-0-is-2.607108-or-260700000-people.}

\subsubsection{C. Modify the plot to include year and food
consumption}\label{c.-modify-the-plot-to-include-year-and-food-consumption}

\includegraphics{ASSIGNMENT1_FINAL_files/figure-latex/unnamed-chunk-8-1.pdf}

\subsection{Question 3}\label{question-3}

\subsubsection{A. Perform a multilinear regression of education and food
consumption for all
countries}\label{a.-perform-a-multilinear-regression-of-education-and-food-consumption-for-all-countries}

\paragraph{All Countries:}\label{all-countries}

Based on the linear regression \(\frac{dN}{Ndt}\ \) = \(2.975*10^{-2}\)
+ \(1.292*10^{-7}\)(Calories (Person/Day)) - \(2.95*10^{-2}\)(Years
Educated), education level is a significant predicator of per capita
population growth rate (p\textless{}0.001). However, food consumption,
defined as the number of calories eaten per person, per day does not
have a significant (p=0.549) effect on \(\frac{dN}{Ndt}\ \).

All other factors remaining constant, for each unit increase in
education level, \(\frac{dN}{Ndt}\ \) significantly decreases by
\(2.95*10^{-2}\).

All things other factors remaining constant, for each unit increase in
calories per person, per day, \(\frac{dN}{Ndt}\ \) increases by
\(1.292*10^{-7}\).

\subsubsection{B. Perform a multilinear regression of education and food
consumption for each
country}\label{b.-perform-a-multilinear-regression-of-education-and-food-consumption-for-each-country}

Linear Regression of Education and Food Consumption Results

Dependent variable:

Per Capita Rate of Change (dN/Ndt)

All

Brazil

India

Japan

Mexico

South Korea

(1)

(2)

(3)

(4)

(5)

(6)

Years of Education

-0.002249***

-0.002420***

-0.001523***

-0.003834***

-0.002280***

-0.002757***

(0.000087)

(0.000343)

(0.000287)

(0.000254)

(0.000260)

(0.000357)

Calories Consumed Per Person Per Day

0.0000001

-0.000001**

-0.0000004

-0.000002**

-0.000005***

-0.0000004

(0.0000002)

(0.000001)

(0.000002)

(0.000001)

(0.000001)

(0.000001)

Constant

0.029747***

0.036311***

0.026005***

0.050606***

0.058878***

0.038582***

(0.000983)

(0.001930)

(0.003791)

(0.002796)

(0.002154)

(0.001146)

Observations

205

41

41

41

41

41

R2

0.770283

0.925976

0.618512

0.915060

0.963831

0.951823

Adjusted R2

0.768008

0.922080

0.598434

0.910590

0.961928

0.949287

Residual Std. Error

0.003665 (df = 202)

0.001333 (df = 38)

0.001320 (df = 38)

0.001327 (df = 38)

0.001305 (df = 38)

0.001285 (df = 38)

F Statistic

338.670700*** (df = 2; 202)

237.673500*** (df = 2; 38)

30.804970*** (df = 2; 38)

204.688500*** (df = 2; 38)

506.316300*** (df = 2; 38)

375.379000*** (df = 2; 38)

Note:

\emph{p\textless{}0.1; \textbf{p\textless{}0.05; }}p\textless{}0.01

\paragraph{Brazil:}\label{brazil-1}

Based on the linear regression \(\frac{dN}{Ndt}\ \) = \(3.631*10^{-2}\)
- \(1.150*10^{-6}\)(Calories (Person/Day)) - \(2.420*10^{-3}\)(Years
Educated), education level is a significant predicator of per capita
population growth rate (p\textless{}0.001). Food consumption, though
less impactful than education, also has a significant (p\textless{}0.05)
effect on \(\frac{dN}{Ndt}\ \).

All other factors remaining constant, for each unit increase in
education level, \(\frac{dN}{Ndt}\ \) decreases by \(2.420*10^{-3}\).

All things other factors remaining constant, for each unit increase in
calories per person, per day, \(\frac{dN}{Ndt}\ \) decreases by
\(1.150*10^{-6}\).

\paragraph{India}\label{india-1}

Based on the linear regression \(\frac{dN}{Ndt}\ \) = \(2.600*10^{-2}\)
- \(4.433*10^{-7}\)(Calories (Person/Day)) - \(1.523*10^{-3}\)(Years
Educated), education level is a significant predicator of per capita
population growth rate (p\textless{}0.001). Food consumption, does not
have a significant (p\textless{}0.05) effect on \(\frac{dN}{Ndt}\ \)
(p=0.793).

All other factors remaining constant, for each unit increase in
education level, \(\frac{dN}{Ndt}\ \) decreases by \(1.523*10^{-3}\).

All things other factors remaining constant, for each unit increase in
calories per person, per day, \(\frac{dN}{Ndt}\ \) decreases by
\(1.523*10^{-3}\).

\paragraph{Japan:}\label{japan}

Based on the linear regression \(\frac{dN}{Ndt}\ \) = \(5.061*10^{-2}\)
- \(1.978*10^{-6}\)(Calories (Person/Day)) - \(3.834*10^{-3}\)(Years
Educated), education level is a significant predicator of per capita
population growth rate (p\textless{}0.001). Food consumption, also has a
significant effect on \(\frac{dN}{Ndt}\ \) (p\textless{}0.05).

All other factors remaining constant, for each unit increase in
education level, \(\frac{dN}{Ndt}\ \) decreases by \(3.834*10^{-3}\).

All other factors remaining constant, for each unit increase in calories
per person, per day, \(\frac{dN}{Ndt}\ \) decreases by
\(1.978*10^{-6}\).

\paragraph{Mexico:}\label{mexico}

Based on the linear regression \(\frac{dN}{Ndt}\ \) = \(5.888*10^{-2}\)
- \(5.306*10^{-6}\)(Calories (Person/Day)) - \(2.280*10^{-3}\)(Years
Educated), education level is a significant predicator of per capita
population growth rate (p\textless{}0.001). Calories eaten per person,
per day simlilarly has a very significant effect on \(\frac{dN}{Ndt}\ \)
(p\textless{}0.001) in Mexico.

All other factors remaining constant, for each unit increase in
education level, \(\frac{dN}{Ndt}\ \) decreases by \(2.280*10^{-3}\).

All things other factors remaining constant, for each unit increase in
calories per person, per day, \(\frac{dN}{Ndt}\ \) decreases by
\(5.306*10^{-6}\).

\paragraph{South Korea:}\label{south-korea}

Based on the linear regression \(\frac{dN}{Ndt}\ \) = \(3.858*10^{-2}\)
- \(4.279*10^{-7}\)(Calories (Person/Day)) - \(2.757*10^{-3}\)(Years
Educated), education level is a significant predicator of per capita
population growth rate (p\textless{}0.001). Calories eaten per person,
per day is not a statistically significant predictor of
\(\frac{dN}{Ndt}\ \) (p=0.0658).

All other factors remaining constant, for each unit increase in
education level, \(\frac{dN}{Ndt}\ \) decreases by \(2.757*10^{-3}\).

All other factors remaining constant, for each unit increase in calories
per person, per day, \(\frac{dN}{Ndt}\ \) decreases by
\(4.279*10^{-7}\).


\end{document}
